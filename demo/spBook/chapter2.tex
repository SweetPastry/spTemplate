\spChapter[贾夫人仙逝扬州城\\冷子兴演说荣国府]{贾夫人仙逝扬州城, 冷子兴演说荣国府}

诗云

一局输赢料不真,香销茶尽尚逡巡。

欲知目下兴衰兆,须问旁观冷眼人。

却说封肃因听见公差传唤,忙出来陪笑启问。那些人只嚷:“快请出甄爷来!”封肃忙陪笑道:“小人姓封,并不姓甄。只有当日小婿姓甄,今已出家一二年了,不知可是问他?”那些公人道:“我们也不知什么‘真’‘假’,因奉太爷之命来问,他既是你女婿,便带了你去亲见太爷面禀,省得乱跑。”说着,不容封肃多言,大家推拥他去了。封家人个个都惊慌,不知何兆。

那天约二更时,只见封肃方回来,欢天喜地。众人忙问端的。他乃说道:“原来本府新升的太爷姓贾名化,本贯胡州人氏,曾与女婿旧日相交。方才在咱门前过去,因见娇杏那丫头买线,所以他只当女婿移住于此。我一一将原故回明,那太爷倒伤感叹息了一回;又问外孙女儿,我说看灯丢了。太爷说:‘不妨,我自使番役务必探访回来。’说了一回话,临走倒送了我二两银子。”甄家娘子听了,不免心中伤感。一宿无话。

至次日,早有雨村遣人送了两封银子,四匹锦缎,答谢甄家娘子,又寄一封密书与封肃,转托问甄家娘子要那娇杏作二房。封肃喜的屁滚尿流,巴不得去奉承,便在女儿前一力撺掇成了,乘夜只用一乘小轿,便把娇杏送进去了。雨村欢喜,自不必说,乃封百金赠封肃,外谢甄家娘子许多物事,令其好生养赡,以待寻访女儿下落。封肃回家无话。

却说娇杏这丫鬟,便是那年回顾雨村者。因偶然一顾,便弄出这段事来,亦是自己意料不到之奇缘。谁想他命运两济,不承望自到雨村身边,只一年便生了一子,又半载,雨村嫡妻忽染疾下世,雨村便将他扶侧作正室夫人了。正是:

偶因一着错,便为人上人。

原来,雨村因那年士隐赠银之后,他于十六日便起身入都,至大比之期,不料他十分得意,已会了进士,选入外班,今已升了本府知府。虽才干优长,未免有些贪酷之弊;且又恃才侮上,那些官员皆侧目而视。不上一年,便被上司寻了个空隙,作成一本,参他“生情狡猾,擅纂礼仪,且沽清正之名,而暗结虎狼之属,致使地方多事,民命不堪”等语。龙颜大怒,即批革职。该部文书一到,本府官员无不喜悦。那雨村心中虽十分惭恨,却面上全无一点怨色,仍是嘻笑自若,交代过公事,将历年做官积的些资本并家小人属送至原籍,安排妥协,却是自己担风袖月,游览天下胜迹。

那日,偶又游至淮扬地面,因闻得今岁鹾政点的是林如海。这林如海姓林名海,表字如海,乃是前科的探花,今已升至兰台寺大夫,本贯姑苏人氏,今钦点出为巡盐御史,到任方一月有馀。原来这林如海之祖,曾袭过列侯,今到如海,业经五世。起初时,只封袭三世,因当今隆恩盛德,远迈前代,额外加恩,至如海之父,又袭了一代;至如海,便从科第出身。虽系钟鼎之家,却亦是书香之族。只可惜这林家支庶不盛,子孙有限,虽有几门,却与如海俱是堂族而已,没甚亲支嫡派的。今如海年已四十,只有一个三岁之子,偏又于去岁死了。虽有几房姬妾,奈他命中无子,亦无可如何之事。今只有嫡妻贾氏生得一女,乳名黛玉,年方五岁。夫妻无子,故爱如珍宝,且又见他聪明清秀,便也欲使他读书识得几个字,不过假充养子之意,聊解膝下荒凉之叹。

雨村正值偶感风寒,病在旅店,将一月光景方渐愈。一因身体劳倦,二因盘费不继,也正欲寻个合式之处,暂且歇下。幸有两个旧友,亦在此境居住,因闻得鹾政欲聘一西宾,雨村便相托友力,谋了进去,且作安身之计。妙在只一个女学生,并两个伴读丫鬟,这女学生年又小,身体又极怯弱,工课不限多寡,故十分省力。

堪堪又是一载的光阴,谁知女学生之母贾氏夫人一疾而终。女学生侍汤奉药,守丧尽哀,遂又将辞馆别图。林如海意欲令女守制读书,故又将他留下。近因女学生哀痛过伤,本自怯弱多病的,触犯旧症,遂连日不曾上学。雨村闲居无聊,每当风日晴和,饭后便出来闲步。

这日,偶至郭外,意欲赏鉴那村野风光。忽信步至一山环水旋,茂林深竹之处,隐隐的有座庙宇,门巷倾颓,墙垣朽败,门前有额,题着“智通寺”三字,门旁又有一副旧破的对联,曰:

身后有馀忘缩手,眼前无路想回头。

雨村看了,因想到:“这两句话,文虽浅近,其意则深。我也曾游过些名山大刹,倒不曾见过这话头,其中想必有个翻过筋斗来的亦未可知,何不进去试试。”想着走入,只有一个龙钟老僧在那里煮粥。雨村见了,便不在意。及至问他两句话,那老僧既聋且昏,齿落舌钝,所答非所问。

雨村不耐烦,便仍出来,意欲到那村肆中沽饮三杯,以助野趣,于是款步行来。将入肆门,只见座上吃酒之客有一人起身大笑,接了出来,口内说:“奇遇,奇遇。”雨村忙看时,此人是都中在古董行中贸易的号冷子兴者,旧日在都相识。雨村最赞这冷子兴是个有作为大本领的人,这子兴又借雨村斯文之名,故二人说话投机,最相契合。

雨村忙笑问道:“老兄何日到此?弟竟不知。今日偶遇,真奇缘也。”子兴道:“去年岁底到家,今因还要入都,从此顺路找个敝友说一句话,承他之情,留我多住两日。我也无紧事,且盘桓两日,待月半时也就起身了。今日敝友有事,我因闲步至此,且歇歇脚,不期这样巧遇!”一面说,一面让雨村同席坐了,另整上酒肴来。二人闲谈漫饮,叙些别后之事。

雨村因问:“近日都中可有新闻没有?”子兴道:“倒没有什么新闻,倒是老先生你贵同宗家,出了一件小小的异事。”雨村笑道:“弟族中无人在都,何谈及此?”子兴笑道:“你们同姓,岂非同宗一族?”雨村问是谁家。子兴道:“荣国府贾府中,可也玷辱了先生的门楣么?”雨村笑道:“原来是他家。若论起来,寒族人丁却不少,自东汉贾复以来,支派繁盛,各省皆有,谁逐细考查得来?若论荣国一支,却是同谱。但他那等荣耀,我们不便去攀扯,至今故越发生疏难认了。”

子兴叹道:“老先生休如此说。如今的这宁荣两门,也都萧疏了,不比先时的光景。”雨村道:“当日宁荣两宅的人口也极多,如何就萧疏了?”冷子兴道:“正是,说来也话长。”雨村道:“去岁我到金陵地界,因欲游览六朝遗迹,那日进了石头城,从他老宅门前经过。街东是宁国府,街西是荣国府,二宅相连,竟将大半条街占了。大门前虽冷落无人,隔着围墙一望,里面厅殿楼阁,也还都峥嵘轩峻;就是后一带花园子里面树木山石,也还都有蓊蔚洇润之气,那里像个衰败之家?”冷子兴笑道:“亏你是进士出身,原来不通!古人有云:‘百足之虫,死而不僵。’如今虽说不及先年那样兴盛,较之平常仕宦之家,到底气像不同。如今生齿日繁,事务日盛,主仆上下,安富尊荣者尽多,运筹谋画者无一;其日用排场费用,又不能将就省俭,如今外面的架子虽未甚倒,内囊却也尽上来了。这还是小事。更有一件大事:谁知这样钟鸣鼎食之家,翰墨诗书之族,如今的儿孙,竟一代不如一代了!”雨村听说,也纳罕道:“这样诗礼之家,岂有不善教育之理?别门不知,只说这宁、荣二宅,是最教子有方的。”

子兴叹道:“正说的是这两门呢。待我告诉你:当日宁国公与荣国公是一母同胞弟兄两个。宁公居长,生了四个儿子。宁公死后,贾代化袭了官,也养了两个儿子:长名贾敷,至八九岁上便死了,只剩了次子贾敬袭了官,如今一味好道,只爱烧丹炼汞,余者一概不在心上。幸而早年留下一子,名唤贾珍,因他父亲一心想作神仙,把官倒让他袭了。他父亲又不肯回原籍来,只在都中城外和道士们胡羼。这位珍爷倒生了一个儿子,今年才十六岁,名叫贾蓉。如今敬老爹一概不管。这珍爷那里肯读书,只一味高乐不了,把宁国府竟翻了过来,也没有人敢来管他。再说荣府你听,方才所说异事,就出在这里。自荣公死后,长子贾代善袭了官,娶的也是金陵世勋史侯家的小姐为妻,生了两个儿子:长子贾赦,次子贾政。如今代善早已去世,太夫人尚在,长子贾赦袭着官,次子贾政,自幼酷喜读书,祖、父最疼,原欲以科甲出身的,不料代善临终时遗本一上,皇上因恤先臣,即时令长子袭官外,问还有几子,立刻引见,遂额外赐了这政老爹一个主事之衔,令其入部习学,如今现已升了员外郎了。这政老爹的夫人王氏,头胎生的公子,名唤贾珠,十四岁进学,不到二十岁就娶了妻生了子,一病死了。第二胎生了一位小姐,生在大年初一,这就奇了;不想后来又生一位公子,说来更奇,一落胎胞,嘴里便衔下一块五彩晶莹的玉来,上面还有许多字迹,就取名叫作宝玉。你道是新奇异事不是?”

雨村笑道:“果然奇异。只怕这人来历不小。”子兴冷笑道:“万人皆如此说,因而乃祖母便先爱如珍宝。那年周岁时,政老爹便要试他将来的志向,便将那世上所有之物摆了无数,与他抓取。谁知他一概不取,伸手只把些脂粉钗环抓来。政老爹便大怒了,说:“‘将来酒色之徒耳!’因此便大不喜悦。独那史老太君还是命根一样。说来又奇,如今长了七八岁,虽然淘气异常,但其聪明乖觉处,百个不及他一个。说起孩子话来也奇怪,他说:‘女儿是水作的骨肉,男人是泥作的骨肉。我见了女儿,我便清爽,见了男子,便觉浊臭逼人。’你道好笑不好笑?将来色鬼无疑了!”雨村罕然厉色忙止道:“非也!可惜你们不知道这人来历。大约政老前辈也错以淫魔色鬼看待了。若非多读书识事,加以致知格物之功,悟道参玄之力,不能知也。”

子兴见他说得这样重大,忙请教其端。雨村道:“天地生人,除大仁大恶两种,余者皆无大异。若大仁者,则应运而生,大恶者,则应劫而生。运生世治,劫生世危。尧、舜、禹、汤、文、武、周、召、孔、孟、董、韩、周、程、张、朱,皆应运而生者。蚩尤,共工,桀,纣,始皇,王莽,曹操,桓温,安禄山,秦桧等,皆应劫而生者。大仁者,修治天下;大恶者,挠乱天下。清明灵秀,天地之正气,仁者之所秉也;残忍乖僻,天地之邪气,恶者之所秉也。今当运隆祚永之朝,太平无为之世,清明灵秀之气所秉者,上至朝廷,下及草野,比比皆是。所馀之秀气,漫无所归,遂为甘露,为和风,洽然溉及四海。彼残忍乖僻之邪气,不能荡溢于光天化日之中,遂凝结充塞于深沟大壑之内,偶因风荡,或被云催,略有摇动感发之意,一丝半缕误而泄出者,偶值灵秀之气适过,正不容邪,邪复妒正,两不相下,亦如风水雷电,地中既遇,既不能消,又不能让,必至搏击掀发后始尽。故其气亦必赋人,发泄一尽始散。使男女偶秉此气而生者,在上则不能成仁人君子,下亦不能为大凶大恶。置之于万万人中,其聪俊灵秀之气,则在万万人之上;其乖僻邪谬不近人情之态,又在万万人之下。若生于公侯富贵之家,则为情痴情种;若生于诗书清贫之族,则为逸士高人,纵再偶生于薄祚寒门,断不能为走卒健仆,甘遭庸人驱制驾驭,必为奇优名倡。如前代之许由、陶潜、阮籍、嵇康、刘伶、王谢二族、顾虎头、陈后主、唐明皇、宋徽宗、刘庭芝、温飞卿、米南宫、石曼卿、柳耆卿、秦少游,近日之倪云林、唐伯虎、祝枝山,再如李龟年,黄幡绰,敬新磨,卓文君,红拂,薛涛,崔莺,朝云之流,此皆易地则同之人也。”

子兴道:“依你说,‘成则王侯败则贼’了。”雨村道:“正是这意。你还不知,我自革职以来,这两年遍游各省,也曾遇见两个异样孩子。所以,方才你一说这宝玉,我就猜着了八九亦是这一派人物。不用远说,只金陵城内,钦差金陵省体仁院总裁甄家,你可知么?”子兴道:“谁人不知!这甄府和贾府就是老亲,又系世交。两家来往,极其亲热的。便在下也和他家来往非止一日了。”

雨村笑道:“去岁我在金陵,也曾有人荐我到甄府处馆。我进去看其光景,谁知他家那等显贵,却是个富而好礼之家,倒是个难得之馆。但这一个学生,虽是启蒙,却比一个举业的还劳神。说起来更可笑,他说:‘必得两个女儿伴着我读书,我方能认得字,心里也明白,不然我自己心里糊涂。’又常对跟他的小厮们说:‘这女儿两个字,极尊贵,极清净的,比那阿弥陀佛,元始天尊的这两个宝号还更尊荣无对的呢!你们这浊口臭舌,万不可唐突了这两个字,要紧。但凡要说时,必须先用清水香茶漱了口才可,设若失错,便要凿牙穿腮等事。’其暴虐浮躁,顽劣憨痴,种种异常。只一放了学,进去见了那些女儿们,其温厚和平,聪敏文雅,竟又变了一个。因此,他令尊也曾下死笞楚过几次,无奈竟不能改。每打的吃疼不过时,他便‘姐姐’‘妹妹’乱叫起来。后来听得里面女儿们拿他取笑:‘因何打急了只管叫姐妹做甚?莫不是求姐妹去说情讨饶?你岂不愧些!’他回答的最妙。他说:‘急疼之时,只叫‘姐姐’妹妹’字样,或可解疼也未可知,因叫了一声,便果觉不疼了,遂得了秘法:每疼痛之极,便连叫姐妹起来了。’你说可笑不可笑?也因祖母溺爱不明,每因孙辱师责子,因此我就辞了馆出来。如今在这巡盐御史林家做馆了。你看,这等子弟,必不能守祖父之根基,从师长之规谏的。只可惜他家几个姊妹都是少有的。”

子兴道:“便是贾府中,现有的三个也不错。政老爹的长女,名元春,现因贤孝才德,选入宫作女史去了。二小姐乃赦老爹之妾所出,名迎春;三小姐乃政老爹之庶出,名探春;四小姐乃宁府珍爷之胞妹,名唤惜春。因史老夫人极爱孙女,都跟在祖母这边一处读书,听得个个不错。”雨村道:“更妙在甄家的风俗,女儿之名,亦皆从男子之名命字,不似别家另外用这些‘春’‘红’‘香’‘玉’等艳字的。何得贾府亦乐此俗套?”子兴道:“不然。只因现今大小姐是正月初一日所生,故名元春,余者方从了‘春’字。上一辈的,却也是从兄弟而来的。现有对证:目今你贵东家林公之夫人,即荣府中赦,政二公之胞妹,在家时名唤贾敏。不信时,你回去细访可知。”雨村拍案笑道:“怪道这女学生读至凡书中有‘敏’字,皆念作‘密’字,每每如是,写字遇着‘敏’字,又减一二笔,我心中就有些疑惑。今听你说的,是为此无疑矣。怪道我这女学生言语举止另是一样,不与近日女子相同,度其母必不凡,方得其女,今知为荣府之孙,又不足罕矣,可惜上月其母竟亡故了。”子兴叹道:“老姊妹四个,这一个是极小的,又没了。长一辈的姊妹,一个也没了。只看这小一辈的,将来之东床如何呢?”

雨村道:“正是。方才说这政公,已有衔玉之儿,又有长子所遗一个弱孙。这赦老竟无一个不成?”子兴道:“政公既有玉儿之后,其妾又生了一个,倒不知其好歹。只眼前现有二子一孙,却不知将来如何。若问那赦公,也有二子,长名贾琏,今已二十来往了,亲上作亲,娶的就是政老爹夫人王氏之内侄女,今已娶了二年。这位琏爷身上现捐的是个同知,也是不肯读书,于世路上好机变,言谈去的,所以如今只在乃叔政老爷家住着,帮着料理些家务。谁知自娶了他令夫人之后,倒上下无一人不称颂他夫人的,琏爷倒退了一射之地:说模样又极标致,言谈又爽利,心机又极深细,竟是个男人万不及一的。”

雨村听了,笑道:“可知我前言不谬。你我方才所说的这几个人,都只怕是那正邪两赋而来一路之人,未可知也。”子兴道:“邪也罢,正也罢,只顾算别人家的帐,你也吃一杯酒才好。”雨村道:“正是,只顾说话,竟多吃了几杯。”子兴笑道:“说着别人家的闲话,正好下酒,即多吃几杯何妨。”雨村向窗外看道:“天也晚了,仔细关了城。我们慢慢的进城再谈,未为不可。”于是,二人起身,算还酒帐。方欲走时,又听得后面有人叫道:“雨村兄,恭喜了!特来报个喜信的。”雨村忙回头看时----